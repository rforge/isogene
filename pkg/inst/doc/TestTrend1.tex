\chapter{Introduction}
\markchapter{Introduction} \label{chap: intro}

Investigation of a dose-response relationship is of primary interest
in many drug-development studies. Typically, in dose-response
experiments the outcome of interest is measured at several
(increasing) dose levels, and the aim  of the analysis is to
establish the form of the dependence of the response on dose
(Agresti 1997). The response can be either the efficacy of a
treatment or the risk associated with the exposure to the treatment
(in toxicology studies). In a typical dose-response study subjects
are randomized to several dose groups, among which there is usually
a control group. Ruberg (1995a, 1995b) and Chuang-Stein and Agresti
(1997) formulated four main questions usually asked in dose-response
studies: (1) Is there any evidence of the drug effect? (2) For which
doses is the response different from the response in the control
group? (3) What is the nature of the dose-response relationship? and
(4) What is the optimal dose?

Within the microarray setting, a dose-response experiment has the
same structure as described above. The response is the
gene-expression at a certain dose level. The dose-response curve,
similarly to the dose-response studies, is assumed to be monotone,
i.e., the gene activity increases or decreases as the dose level
increases. The direction of the relationship is usually unknown in
advance.

In this chapter we focus on the first question: is there any
evidence of the drug effect? To answer this question, we test for
the null hypothesis of homogeneity of means (no dose effect) against
an ordered alternative. We compare several testing procedures, that
take into account the order restriction of the means with respect to
the increasing doses and that adjust for multiple testing. In
particular, we discuss the testing procedures of Williams (Williams
1971 and 1972), Marcus (Marcus 1976), the global likelihood ratio
test ($LRT$, Barlow \textit{et al.}\ 1972, and Robertson \textit{et
al.}\ 1988), and the $M$ (Hu \textit{et al.}\ 2005) statistic.
Moreover, we propose a novel procedure based on a modification of
the estimator of standard error of the $M$ statistic.

Williams (1971, 1972) proposed a step-down procedure to test for the
dose effect. The tests are performed sequentially from the
comparison between the isotonic mean of the highest dose and the
sample mean of the control, to the comparison between the isotonic
mean of the lowest dose and the sample mean of the control. The
procedure stops at the dose level where the null hypothesis (of no
dose effect) is not rejected. Marcus (1976) proposed a modification
of the Williams procedure, in which the sample mean of the control
was replaced by the isotonic mean of the control. A global
likelihood ratio test, discussed by Bartholomew \textit{et al.}\
(1961), Barlow \textit{et al.}\ (1972), and Robertson \textit{et
al.}\ (1988), uses the ratio between the variance calculated under
the null hypothesis and the variance calculated under an ordered
alternative. Recently, Hu \textit{et al.}\ (2005) proposed a test
statistic that was similar to Marcus' statistic, but with the
variance estimator calculated under the ordered alternative. The
degrees of freedom of the $M$ statistic (the difference between the
number of observations and the number of dose levels) were fixed for
all the genes and all the arrays. We propose a modification for the
variance estimator of the $M$ statistic. Namely, the difference
between the number of observations and the unique number of isotonic
means is used as the degrees of freedom for the variance estimator.

\texttt{IsoGene} is an R package for the analysis of dose-response microarray experiments. The package can be used in order to identify differentially expressed genes, which are, within the framework of dose-response microarray experiments, a subset of genes, for which a monotone relationship between the gene expression and doses can be detected. Inference is based on resampling methods (both permutations and the significance analysis of Microarray (SAM)), in which the multiplicity issue is addressed by adjustment techques controlling for the false discovery rate (FDR). This guide provides a tutorial to the features of the package. It illustrates the capability of the \texttt{IsoGene} library and provides some background information about the methodology used for the analysis. In Chapter 2, we review different testing procedures; while in Chapter 3, we illustrate how the methodology discussed in Chapter 2 can be implemented using the \texttt{IsoGene} library.


